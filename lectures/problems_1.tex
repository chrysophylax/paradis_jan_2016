\documentclass[12pt]{hitec}

\usepackage{fancyvrb}
\usepackage{hyperref}

\title{Problem set 1}
\author{Joe Armstrong}

\begin{document}

\maketitle

\tableofcontents

\section{Introduction}
These problems are for lectures E1 and E2.

Additional material can be found in the course book
{\sl Programming Erlang, 2'nd edition}.

\section{Problems}

The first exercises involve correcting
incorrect programs. The program \verb+bad_prog1.erl+ in my github repository
\newline
\href{https://github.com/joearms/paradis\_jan\_2016/blob/master/lectures/bad\_prog1.erl}
     {https://github.com/joearms/paradis\_jan\_2016/}

has a 
large number of errors.

\VerbatimInput[frame=single]{./bad_prog1.erl}

Correct all the errors that you can find.  Write unit
tests\footnote{See the note on unit tests.}  for this module and run
these tests.

See how many errors you can find {\sl before} running this through the
compiler.

Hint: Read the error messages - correct the first error and recompile
before tackling the second error. Having fixed the first error the
error messages that the compiler generates may change of subsequent
recompilation.

As you get experienced look at only the line number in the error
message then try to see the error in your code at this line without
reading the remainder of the diagnostic.

Note: Good programs should have good documentation and
comments. Correct the comments as well as the code, and document the
module in an appropriate manner.



\section{A bank account}

The following program implements a simple bank process, where the code
you have to write has been replaced by the atom \verb+implement_this+.

Replace the \verb+implement_this+ atoms by correct code. 

\VerbatimInput[frame=single]{./bank.erl}

When your program works the unit test \verb+bank:test()+ should
evaluate to \verb+horray+

Look at the code in \verb+e1.erl+ for inspiration.

\section{Advanced: Many bank account}

The previous bank account was for one person only. Implement a module
\verb+multi_bank.erl+ that handles bank accounts for many
people. You'll have to add an additional argument \verb+Who+ to all
the fucntions in the API, so for example where \verb+bank.erl+ exports
\verb+add(Pid, X)+ \verb+multi_bank.erl+ would have a function
\verb+add(Pid, Who, X)+ where \verb+Who+ is a person name (for example, the
atom \verb+joe+).

Add an additional function \verb+lend(Pid, From, To, Ammount)+ to
transfer money between bank accounts of the people \verb+From+ and \verb+To+.

\section{Unit Tests}

Here's an example of a simple module, containing a test function \verb+test()+.

\VerbatimInput[frame=single]{./math0.erl}

The unit tests are in the body of the function \verb+test()+. To run the tests
we proceed as follows:

\begin{verbatim}
> erl
Eshell V7.1  (abort with ^G)
1> c(math0).
{ok,math0}
2> math0:test().
horray
\end{verbatim}

Here we have used pattern matching to test the return values of the function.
The line:
\begin{verbatim}
24 = fac(4)
\end{verbatim}

will fail with a pattern matching error if \verb+fac(4)+ does not return  \verb+24+
(which is the value of factorial 4.

If we call \verb+fac(N)+ with an argument that is not an integer the function
should raise an exception. We can test that it has throw an exception with a
statement like:

\begin{verbatim}
  {'EXIT', _} = (catch fac(X))
\end{verbatim}

In a later lecture I'll talk about catching exceptions, but for now all you need
to know is that \verb+catch(F(X))+ traps errors occurring during the evaluation
of \verb+F(X)+ converting them to error tuples of the form \verb+{'EXIT', Why}+
which we match with the pattern \verb+{'EXIT', _}+.

\section*{Notes}

\href{https://github.com/joearms/paradis\_jan\_2016/blob/master/lectures/}
     {https://github.com/joearms/paradis\_jan\_2016/blob/master/lectures/}
     contains material referred to in the problems.
     
If you want to build the lectures locally you will need a working verison
of erlang and pdflatex installed on your machine.

\end{document}

