\documentclass[12pt]{hitec}

\usepackage{fancyvrb}
%% \usepackage{hyperref}

\title{Problem Set 3}
\author{Joe Armstrong}
\date{}

\begin{document}

\maketitle

\tableofcontents

\section{Introduction}
These problems are for lectures E5 and E6.

Additional material can be found in Chapters 11 to 14 of the course book
{\sl Programming Erlang, 2'nd edition}.

\subsection*{On ambiguities in the questions}

These questions might appear to you to be unclear or to have multiple
interpretations.

If you think the question is unclear and has several interpretations,
then choose one of the interpretations and and answer the question
with the chosen interpretation.

Include in your answer a description of why you thought the question
was unclear and which of the possible interpretations you have
chosen. You might get extra credits for this.

If you think the question is crazy and cannot be answered then write a
description of why you cannot answer it.

When you get a job at the end of your courses and start programming
you will find that specifications are almost always unclear -- this is very
common -- in these circumstances there is often nobody to ask so have
to choose an interpretation of the specification, write down your
assumptions and solve the problem subject to your interpretation.
 
\section{Problems}

This week I've put the problems into three categories:

\begin{itemize}
\item {\bf normal} solve these to get {\sl Godk\"{a}nd}.
\item {\bf advanced} solve these to get {\sl V\"{a}l Godk\"{a}nd}.
\item {\bf optional} solve these to get a deeper understanding of the problem. You don't
have to hand in your solutions to these, but solving them, or attempting them will
deepen your understanding of Erlang.
\end{itemize}

The sections and subsections in this paper are marked appropriately.

\section{Simple Processes}


Write the following routines:

\subsection{normal: \texttt{double:start()}}

\verb+double:start()+ creates a registered process called double.
If you send it a message containing an integer it should
double the integer and print the result.

If it is sent a non-integer it should print an error message
and crash.

Start the double process and test that it works correctly.

\subsection{normal: \texttt{monitor:start()}}

\verb+monitor:start()+ should create a processes that starts and
then monitors the double process that you made in the last section.
 
If the monitor process detects that the double process has crashed
it should print an error message and start a new double process.

Start the monitor process. Send the double process something that causes it
to crash. Then check that a new double process has been created and that
the system has recovered from the error.

\section{advanced: Reliable Client-Server}

Here we assume that the server might crash. If a client sends a message to
and does not get a reply (this is detected by a timeout) the client
waits a random time and tries again. It tries this strategy several times
and gives up if the cannot get a reply from the server after 10 tries. 

\begin{itemize}

\item Create a registered server called double.

\item If you send it an integer it doubles it and sends back the reply.

\item It crashes if you send it an atom.

\item Make a process that sleeps for a random time and sends a message
  to the double server that causes it to crash.

\item Make a monitor process that detects that the server has
  crashed. It restarts the server after a random delay.

\item Make a client function that sends a request to the server and
  times out if the request is not satisfied. We can assume the server
  has crashed. The client should wait a second and then try again.

\item  Abort the client if it has tried more than ten times.
\end{itemize}

Hints: \verb+crypto:rand_uniform(Lo, Hi) -> N+
Generates a random number \verb+N+, where \verb+ Lo =< N < Hi+.

A sleep function can be defined as follows:

\begin{Verbatim}
sleep(T) ->
    %% sleep for T milliseconds
    receive
      after T ->
        true
    end.
\end{Verbatim}

\section{Parallel Maps}

\verb+smap(F, L)+ (sequential map) maps a function \verb+F+ over a
list of items \verb+L+  can be defined as follows:

\begin{Verbatim}
smap(_, [])    -> [];
smap(F, [H|T]) -> [F(H) | smap(F, T)].
\end{Verbatim}

A parallel version \verb+pmap(F,L)+ can be written as follows:
\begin{Verbatim}
pmap(F, L) -> 
    S = self(),
    Pids = smap(fun(I) -> 
                       spawn(fun() -> do_f(S, F, I) end)
               end, L),
    %% gather the results
    gather(Pids).

do_f(Parent, F, I) ->                                          
    Parent ! {self(), (catch F(I))}.

gather([Pid|T]) ->
    receive
        {Pid, Ret} -> [Ret|gather(T)]
    end;
gather([]) ->
    [].
\end{Verbatim}

There, are many varients on this simple pattern, so in these exercises
you will write a number of parallel mapping functions.

\subsection{normal: pmap\_max(F, L, MaxWorkers)}

Write a function \verb+pmap_max(F, L, MaxWorkers)+ that produces the same return
value as \verb+smap(F, L)+ by evaluating up to \verb+MaxWorkers+ elements of the list
in parallel. Assume there are no exceptions raised when evaluating \verb+F+.

\subsection{normal: pmap\_any(F, L)}

Write a function \verb+pmap_any(F, L)+ that maps \verb+F+ over the
list \verb+L+ in parallel but where the items in the returned list
occur in the order they were computed.

Assume that:

\begin{Verbatim}
fib(1) -> 1;
fib(2) -> 1; 
fib(N) -> fib(N-1) + fib(N-2).
\end{Verbatim}

Then:

\begin{Verbatim}
> pmap_any(fun fib/1, [35,2,20]).
[1,6765,9227465]
\end{Verbatim}

Why is this? \verb+fib(2)+ is computed quickly so is first in the
list.  \verb+fib(35)+ comes last since it takes a long time to
compute. If we compute everything in parallel we should get the answer
to \verb+fib(2)+ first.

\subsection{normal: pmap\_any\_tagged(F, L)}

This is similar to the last exercise, with the addition that the returned
list should contain a copy of the input item that was processed. For example:

\begin{Verbatim}
> pmap_any_tagged(fun fib/1, [35,2,20]).
[{2,1},{20,6765},{35,9227465}]
\end{Verbatim}

\subsection{normal: pmap\_any\_tagged\_max\_time(F, L, MaxTime)}

This is similar to the last exercise, but with a maximum total time
(\verb+MaxTime+) by which all parallel computations must
terminate. \verb+MaxTime+ is a time in milliseconds.  All on-going
computations what have not terminated within time \verb+MaxTime+ must
be killed. This means \verb+pmap_any_tagged_max_times+ will always
return within \verb+MaxTime+ milliseconds.

For example:

\begin{Verbatim}
> pmap_any_tagged_maxtime(fun fib/1, [35,2,456,20],2000).
[{2,1},{20,6765},{35,9227465}]
\end{Verbatim}

Note that the computation of \verb+fib(456)+ takes a very long time,
and was aborted.

{\sl Aside: How long would it take to calculate fib(456) using the above program?
Try estimating this. Can you think of a much faster way to compute fib(N)?}

\subsection{advanced: pmap\_timeout(F, L, Timeout, MaxWorkers)}

This map function returns the elements in the list in the same order
as the original list. The arguments should be computed in parallel. If
a computation fails, the return value in the list should be the atom
\verb+error+.

If the the computation of an individual element takes more that
\verb+Timeout+ milliseconds then the computation should be aborted and
the return value in the list should be the atom \verb+timeout+.

No more than at most \verb+MaxWorkers+ processes should run in parallel.

For example:

\begin{Verbatim}
> pmap_timeout(fun fib/1, [10,20,2000,abc], 2000, 3).
[55,6765,timeout,error]
\end{Verbatim}

The above call means ``compute \verb+fib(I)+ for the elements \verb+I+
in the list \verb+[10,20,200,abc]+. Return \verb+timeout+ if the
computation of \verb+fib(I)+ takes more than 2 seconds, \verb+error+
if the computation of \verb+fib(I)+ fails and allow no more than 3
computations of \verb+fib(I)+ to run in parallel.''

\subsection{advanced: pmap\_maxtime(F, L, MaxTime, MaxWorkers)}

This is similar to the last exercise, but in this case \verb+MaxTime+ is to
be interpreted as the {\sl Total} time for all computations to take.

\section{Sending messages in a ring}

\subsection{normal: \texttt{ring:start(N,M)}}

\verb+ring:start(N,M)+ should create a ring of \verb+N+ processes,
then send an integer \verb+M+ times round the ring. The message should
start as the integer 0 and each process in the ring should increment
the integer by 1.  After the message has been round the ring \verb+M+
times its final value should be \verb+N*M+. 

\verb+ring:start(N,M)+ should return the value of the last message
(which should be \verb+N*M+) and make sure that all the processes in
the ring have terminated.

\subsection{advanced: \texttt{ring:time(N,M)}}

Time how long it takes to run \verb+ring:start(N,M)+\\
(Hint, use
\verb+timer:tc(Mod, Func, Args)+\footnote{Consult the documentation for
  the timer module.}).

Measure the time it takes to send a message round the ring for
reasonably large values of \verb+N+ and \verb+M+. The product
of \verb+N*M+ should be a few million. Try this for tens of thousands to
millions of processes. Experiment with the \verb+erl -smp disable+.

\section{optional: A Mail Server}

Write a simple mail server. The server should respond to 4 messages:

\begin{description}

\item \verb+{message,From,Pwd,To,Subject,Data}+\\
  send a message to
  the server, all the arguments are binaries. When the message arrives
  at the server the server should check that the user \verb+From+ has
  password \verb+Pwd+ and that the user \verb+To+ exists. 

  The possible responses are:
  
  \begin{itemize}
  
  \item \verb+eNoSuchUser+ -- if the person \verb+To+ does not exist.
  
  \item \verb+eBadPassword+  -- if the person \verb+From+ does not have the
  password \verb+Pwd+.
  
\item \verb+{ok, MailId}+ -- if the mail was accepted. \verb+MailId+ is
  a unique identifier identifying this mail.
  The mail should be stored in a mailbox 
  associated with the user \verb+To+ together with a timestamp.
 

 \end{itemize}
  
 
\item \verb+{list_my_mail,User,Pwd}+\\
  This message is used to check for mail. 

  The possible responses are:
  
  \begin{itemize}
  
  \item \verb+eNoSuchUser+ -- if the person \verb+To+ does not exist.
  
  \item \verb+eBadPassword+  -- if the person \verb+From+ does not have the
  password \verb+Pwd+.
  
\item \verb+{ok, [{MailId,From,Header}]}+ -- returns a list of the mail
  waiting for the the user.
  \end{itemize}

 
\item \verb+{get_mail,MailId,Pwd}+\\
  This message is used to fetch a specific mail. \verb+Pwd+ is  
  the password of the user to whom the mail with Id \verb+mailId+ was sent.

  The possible responses are:
  
  \begin{itemize}
  
  \item \verb+error+ -- the mail Id does not exist or the password is incorrect.
  
\item \verb+{ok, [{MailId,From,Header,Content,Timestamp}]}+ -- returns the mail
  \end{itemize}

 
\item \verb+{delete_mail,MailId,Pwd}+\\
  This message is used to delete a specific mail. The arguments have the same
  meaning as for the \verb+get_mail+ command.

  The possible responses are:
  
  \begin{itemize}
  
  \item \verb+error+ -- the mail Id does not exist or the password is incorrect.
  \item \verb+{deleted, MailId}+ the mail was deleted.
    
  \end{itemize}

  

\end{description}

\subsection{A sample session with the mail server}

Assume Alice has password \verb+ecila+  and Bob \verb+blurb+

\begin{enumerate}

\item {\sl Alice sends a message to Bob}. She sends the message:

\begin{verbatim}
{mail,<<"alice">>,<<"ecila">>,<<"bob">>,
                 <<"hello">>,<<"This is ..">>)
\end{verbatim}

The mail server check that Alice's password is correct. It is correct so the
mail is added to Jim's mailbox. The server responds \verb+{ok,43}+. This is mail
number 43. The server adds a time stamp to the mail to record when it received
the mail.

\item {\sl Bob want to check his mail}. He sends the message:
 
\verb+{check_my_mail,<<"bob">>,<<"blurb">>>}+

To the server. Bob has supplied the correct password, so the server responds:

\verb+{mails, [..., {mail,43,<<"alice">>,<<"hello">>}, ...]}+

So Bob can see that he has got mail from Alice. 

\item Bob retrieves the mail by sending the message:

\verb+{get_mail,43,<<"blurb">>}+ 

to the server. He wants to read mail 43 
which the server knows is in Bob's mailbox. To do this Bob supplies his password
(\verb+blurb+). The password is correct so the mail server replies:

\begin{verbatim}
{msg,43,<<"alice">>,
        <<"hello">>,<<"This is ...">>,
        <<"2013-02-12 12:12:34">>}
\end{verbatim}



Bob reads the mail.

\item Bob deletes the mail by sending the message:

\verb+{delete_mail,43,<<"blurb">>}+ to the server. The server responds
\verb+{deleted,43}+

\end{enumerate}



\end{document}
