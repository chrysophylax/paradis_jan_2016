\documentclass[12pt]{hitec}

\usepackage{fancyvrb}
%% \usepackage{hyperref}

\title{Assigments week 4}
\author{Joe Armstrong}

\begin{document}

\maketitle

\tableofcontents

\section{Introduction}
These problems are for lectures E7 and E8.

\section{Problems}

The problems into two categories:

\begin{itemize}
\item {\bf normal} solve these to get {\sl Godk\"{a}nd}.
\item {\bf advanced} solve these to get {\sl V\"{a}l Godk\"{a}nd}.
\end{itemize}

The sections and subsections in this paper are marked appropriately.

\section{Introduction}

This week we'll make a complete program, sewing all the bits we're learned together.
We'll make a file sharing program, similar to {\sl bit torrent}, though far simpler.

Before we do the exercises I should explain in general terms how bit torrent works.

\section*{How Bit Torrent Works}

Bit Torrent works by having three different programs and three different protocols.

\begin{itemize}

\item The {\bf Tracker} keeps  track of of a number of peers. A peer can offer a file
  meaning that it has the file and will allow other peers to obtain a copy of the file.
\item A {\bf Peer} is a program that either offers a file or wants a file
\item An {\bf Index} where files and their MD5 checksums are published.
  
\end{itemize}

Three protocols are involved:

\begin{itemize}
  
\item {\sl The Peer - Tracker protocol}. This protocol has two messages:
  \begin{itemize}

  \item
    The peer can send an \verb+{i_am_interested_in, Md5}+ message to
    the tracker.  This means that the peer is interested in the file
    whoes MD5 checksum is \verb+Md5+ Interested means it has parts of
    the file and wants to share them, or has non of the file.  The
    Tracker should store the IP address and Port of the peer
    concerned.

  \item A peer can send a \verb+{who_is_interested,Md5}+ message to the
    server. The server will reply to this with the address of the
    machines who are interested in the file.

  \end{itemize}

\item {\sl The Peer - Peer protocol}. This protocol has two messages:

  \verb+{what_have_you, Md5}+ Is a message sent by one peer to another, asking
  about the file it wants to obtain. The reply is a message,
  \verb+{i_have,[{Start,Stop}]}+ containing tuples
  describing which parts of the file the peer has.
  
  \verb+{send_me,MD5,Start,Stop}+ the peer who wants a file sends a
  messaging asking for part of the file with checksum \verb+Md5+.
  The receiving client send the part of the file concerned;

\item {\sl The Peer - Index Protocol} - so far the Tracker knows
  nothing about the file being tracked, only the MD5
  checksum. Somewhere we need to keep an index of the File names and
  their corresponding MD5 checksums and sizes. This is the job of the
  index.

  A peer that want to share a file computes the \verb+Md5+ checksum of the file
  and sends a \verb+{add_to_index, FileName, Md5, Size}+ message to the index,
  and a \verb+{i_have, Md5}+ message to the tracker.

  The index publishes a list of all files it knows about together with the
  MD5 checksums of the files and the file sizes. This is what the Pirate Bay did.
  
\end{itemize}

\section{Normal: A Basic Server}

Write three programs \verb+tracker.erl+, \verb+index.erl+ and \verb+peer.erl+
which implement a complete Bit-torrent like system. Run this in a single Erlang node.

Write the tracker and and index programs as \verb+gen_server+ plugins.
Read \verb+http://learnyousomeerlang.com/what-is-otp+ for more details
on the gen server (or my book!).

Write some test routine to see that everything works, and documentation.

To simplify the problem assume the files are small and that the entire file
can be transferred in one message.

\section{Advanced: Parallel Transfer}

A Peer can discover that the file they are interest in can be hosted
by more than one other Peers. Your program can be made more robust and
faster by requesting different parts of the file in parallel.

Implement this.

\section{Advanced: An TCP Server}

In the previous section you implemented the system on one node using
Erlang messages.  Modify the code to use TCP as the transport
medium. Note: in the first section to notion of an ``address'' could
be interpreted as meaning ``A process Identifier'' - for this section
we interpreted an address as meaning ``Ad IP adress and Port''.


\section*{Note}

The real bit torrent is more complicated than this, and uses a ``tit
for tat'' algorithm.  A peer that has the entire file is called {\sl a
  seed} - peers that do not have the entire file are discouraged from
downloding from the seed, but encouraged to find another peer and get
the data from them instead of bothering the seed. This is on a basis
of trust. Peers that send you data are rewarded by you sending them
data, and so on.

In our problems the entire file has a checksum - but if we request a
smaller fragment of the file there is no checksum for the fragment so
we will only discover if the file is corrupt after obtaining all the
fragments.

In the real Bit Torrent a file is represented by a list of checksums
\verb+[Md51, Md52, Md53, ....]+ which are the checksums of the 1'st MB
block, the 2'md MB block and so on\footnote{It's not actuall 1 MByte - the
  size varies depending upon the total size of the file}.  A file containing a list of
checksums\footnote{And a few other things, like the total size of the file ...}
is called {\sl a torrent} and the checksum of the file
containing a list of torrents is the torrent identifier.

\end{document}

